\documentclass{article}

\usepackage[left=1in, right=1in]{geometry}
\usepackage{amsmath,amssymb}

\title{Weak forms for Bloch eigenproblems}
\author{Connor D. Pierce}
\date{\today}


% Customizable math macros
\newcommand{\imag}{\iota}
\newcommand{\vect}[1]{\boldsymbol{#1}}
\newcommand{\x}{x}
\newcommand{\vecx}{\vect{\x}}
\newcommand{\wavenum}{\kappa}
\newcommand{\wavevec}{\vect{\wavenum}}
\newcommand{\bloch}{e^{-\imag \wavevec \cdot \vecx}}
\newcommand{\blochcomp}[1]{e^{-\imag \wavenum_{#1} \x_{#1}}}
\newcommand{\kernel}[1]{\tilde{#1}}
\newcommand{\dx}{\,\mathrm{d}x}
\newcommand{\ds}{\,\mathrm{d}s}


\begin{document}
\maketitle
\tableofcontents


%%%%%%%%%%%%%%%%%%%%%%%%%%%%%%%%%%%%%%%%%%%%%%%%%%%%%%%%%%%%%%%%%%%%%%%%%%%%%%%%
\section{Notation}

In this document, the following notation is observed:
%
\begin{description}
    \item[$\omega$] temporal frequency
    \item[$\wavenum$] wave number (spatial frequency)
    \item[$\imag$] (``iota'') the imaginary unit
    \item[$i,j,k,l,m$] subscripts indicating vector/tensor components
\end{description}
%
Bold symbols indicate vectors, e.g. \(\vecx\) is the position vector with components \(\x_{i}\).


%%%%%%%%%%%%%%%%%%%%%%%%%%%%%%%%%%%%%%%%%%%%%%%%%%%%%%%%%%%%%%%%%%%%%%%%%%%%%%%%
\section{Problems considered} \label{sec:prob}


%%%%%%%%%%%%%%%%%%%%%%%%%%%%%%%%%%%%%%%%%%%%%%%%%%%%%%%%%%%%
\subsection{Scalar Helmholtz equation} \label{sec:prob.scalar}

Eigenproblem:
%
\begin{align}
    %
    \nabla \cdot \left(
    \boldsymbol{E}(\vecx) \nabla w(\vecx, t)
    \right) =& \rho(\vecx) \ddot{w}(\vecx, t) \nonumber \\
    %
    \left(
    E_{ij}(\vecx) w_{,j}(\vecx, t)
    \right)_{,i} =& \rho(\vecx) \ddot{w}(\vecx, t) \nonumber \\
    %
    w(\vecx, t) =& u(\vecx) e^{-\imag \omega t} \nonumber \\
    %
    \left(
        E_{ij}(\vecx) u_{,j}(\vecx)
    \right)_{,i} e^{-\imag \omega t} =& -\omega^2 \rho(\vecx) u(\vecx) e^{
        -\imag \omega t
    } \nonumber \\
    %
    \left(
        E_{ij}(\vecx) u_{,j}(\vecx)
    \right)_{,i} =& -\omega^2 \rho(\vecx) u(\vecx) \label{eq:prob.scalar}
\end{align}


%%%%%%%%%%%%%%%%%%%%%%%%%%%%%%%%%%%%%%%%%%%%%%%%%%%%%%%%%%%%
\subsection{Elasticity equation} \label{sec:prob.vector}

Eigenproblem:
%
\begin{align}
    \nabla \cdot \left(
        \boldsymbol{E}(\vecx) \nabla_s \vect{w}(\vecx, t)
    \right) =& \rho(\vecx) \ddot{\vect{w}}(\vecx, t) \nonumber \\
    %
    \left(
        E_{ijkl}(\vecx) \frac{1}{2} \left(w_{k,l}(\vecx, t) + w_{l,k}(\vecx, t)\right)
    \right)_{,i} =& \rho(\vecx) \ddot{w}_j(\vecx, t) \nonumber \\
    %
    w_j(\vecx, t) =& u_j(\vecx) e^{-\imag \omega t} \nonumber \\
    %
    \left(
        E_{ijkl}(\vecx) \frac{1}{2} \left(u_{k,l}(\vecx) + u_{l,k}(\vecx)\right)
    \right)_{,i} e^{-\imag \omega t} =& -\omega^2 \rho(\vecx) u_j(\vecx) e^{
        -\imag \omega t
    } \nonumber \\
    %
    \left(
        E_{ijkl}(\vecx) \frac{1}{2}\left(u_{k,l}(\vecx) + u_{l,k}(\vecx)\right)
    \right)_{,i} =& -\omega^2 \rho(\vecx) u_j(\vecx) \label{eq:prob.vector}
\end{align}


%%%%%%%%%%%%%%%%%%%%%%%%%%%%%%%%%%%%%%%%%%%%%%%%%%%%%%%%%%%%
\subsection{Bloch ansatz} \label{sec:prob.bloch-ansatz}

Bloch waves have the form
%
\begin{align}
    u(\vecx) =& \kernel{u}(\vecx) \bloch \, , \label{eq:prob.bloch-ansatz}
\end{align}
%
where \(\kernel{u}(\vecx)\) is a function that is periodic on the unit cell.


%%%%%%%%%%%%%%%%%%%%%%%%%%%%%%%%%%%%%%%%%%%%%%%%%%%%%%%%%%%%%%%%%%%%%%%%%%%%%%%%
\section{Direct ($\wavevec(\omega)$)} \label{sec:direct}

In the direct formulation (see Collet et al., 2011), we prescribe the temporal frequency \(\omega\) and wave direction \(\vect{\phi}\) and compute the wavevector magnitudes as the eigenvalues. The periodic functions \(\kernel{u}(\vecx)\) are the eigenfunctions.

\noindent TODO: write me


%%%%%%%%%%%%%%%%%%%%%%%%%%%%%%%%%%%%%%%%%%%%%%%%%%%%%%%%%%%%%%%%%%%%%%%%%%%%%%%%
\section{Indirect ($\omega(\wavevec)$), untransformed} \label{sec:indirect}

In the untransformed indirect formulation, we prescribe the wavevector through Floquet-periodic boundary conditions and solve for the temporal frequencies \(\omega\) as eigenvalues. The Floquet-periodic displacement fields \(u(\vecx)\) are the corresponding eigenfunctions.

\noindent TODO: write me


%%%%%%%%%%%%%%%%%%%%%%%%%%%%%%%%%%%%%%%%%%%%%%%%%%%%%%%%%%%%%%%%%%%%%%%%%%%%%%%
\section{Indirect ($\omega(\wavevec)$), with Bloch transformation}
\label{sec:it}

In the transformed indirect formulation, the Bloch ansatz is used to transform the equations of motion in terms of the periodic function \(\kernel{u}(\vecx)\). The wavevector \(\wavevec\) is prescribed through additional terms that arise in the strong form of the problem, and the temporal frequency \(\omega\) is solved as the eigenvalue. The periodic functions \(\kernel{u}(\vecx)\) are the corresponding eigenfunctions.


%%%%%%%%%%%%%%%%%%%%%%%%%%%%%%%%%%%%%%%%%%%%%%%%%%%%%%%%%%%%
\subsection{Scalar Helmholtz equation} \label{sec:it.scalar}

%%%%%%%%%%%%%%%%%%%%%%%%%%%%%%%%%%%%%%%%%%%%%%%%%%%%%%%%%%%%
\subsubsection{Derivation of transformed strong form}
\label{sec:it.scalar.strong-derivation}

Consider the eigenproblem (\ref{eq:prob.scalar}) and insert the Bloch ansatz (\ref{eq:prob.bloch-ansatz}):
%
\begin{align}
    \left(
        E_{ij}(\vecx) \left[\kernel{u}(\vecx) \blochcomp{k}\right]_{,j}
    \right)_{,i} &=
    -\omega^2 \rho(\vecx) \kernel{u}(\vecx) \blochcomp{k}
    \nonumber \\
    %
    \left(
        E_{ij}(\vecx) \left[
            \kernel{u}_{,j}(\vecx) \blochcomp{k}
            + \kernel{u}(\vecx) (\blochcomp{k})_{,j}
        \right]
    \right)_{,i} &=
    -\omega^2 \rho(\vecx) \kernel{u}(\vecx) \blochcomp{k}
    \nonumber \\
    %
    \left(
        E_{ij}(\vecx) \left[
            \kernel{u}_{,j}(\vecx) \blochcomp{k}
            + \kernel{u}(\vecx) \blochcomp{m} (-\imag \wavenum_{k} \delta_{kj})
        \right]
    \right)_{,i} &=
    -\omega^2 \rho(\vecx) \kernel{u}(\vecx) \blochcomp{k}
    \nonumber \\
    %
    \left(
        E_{ij}(\vecx) \left[
            \kernel{u}_{,j}(\vecx) \blochcomp{k}
            - \imag \wavenum_{j} \kernel{u}(\vecx) \blochcomp{k}
        \right]
    \right)_{,i} &=
    -\omega^2 \rho(\vecx) \kernel{u}(\vecx) \blochcomp{k}
    \nonumber \\
    %
    \left(
        E_{ij}(\vecx) \kernel{u}_{,j}(\vecx) \blochcomp{k}
        - \imag E_{ij}(\vecx) \wavenum_{j} \kernel{u}(\vecx) \blochcomp{k}
    \right)_{,i} &=
    -\omega^2 \rho(\vecx) \kernel{u}(\vecx) \blochcomp{k}
    \nonumber \\
    %
    \left(
        E_{ij}(\vecx) \kernel{u}_{,j}(\vecx) \blochcomp{k}
    \right)_{,i}
    - \imag \left(
        E_{ij}(\vecx) \wavenum_{j} \kernel{u}(\vecx) \blochcomp{k}
    \right)_{,i} &=
    -\omega^2 \rho(\vecx) \kernel{u}(\vecx) \blochcomp{k}
    \nonumber \\
    %
    \left(
        \left[
            E_{ij}(\vecx) \kernel{u}_{,j}(\vecx)
        \right]_{,i} \blochcomp{k}
        -\imag \wavenum_{i} E_{ij}(\vecx) \kernel{u}_{,j}(\vecx) \blochcomp{k}
    \right)& \nonumber \\
    - \imag \left(
        \left[
            E_{ij}(\vecx) \wavenum_{j} \kernel{u}(\vecx)
        \right]_{,i} \blochcomp{k}
        - \imag \wavenum_{i} E_{ij}(\vecx) \wavenum_{j} \kernel{u}(\vecx)
        \blochcomp{k}
    \right) &=
    -\omega^2 \rho(\vecx) \kernel{u}(\vecx) \blochcomp{k}
    \nonumber \\
    %
    \left(
        \left[
            E_{ij}(\vecx) \kernel{u}_{,j}(\vecx)
        \right]_{,i}
        -\imag \wavenum_{i} E_{ij}(\vecx) \kernel{u}_{,j}(\vecx)
    \right)& \nonumber \\
    - \imag \left(
        \left[
            E_{ij}(\vecx) \wavenum_{j} \kernel{u}(\vecx)
        \right]_{,i}
        - \imag \wavenum_{i} E_{ij}(\vecx) \wavenum_{j} \kernel{u}(\vecx)
    \right) &=
    -\omega^2 \rho(\vecx) \kernel{u}(\vecx)
\end{align}

\subsubsection{Transformed strong forms} \label{sec:it.scalar.strong}

\paragraph{General case}
%
\begin{align}
    \left(
        \left[ E_{ij}(\vecx) \kernel{u}_{,j}(\vecx) \right]_{,i}
        - \wavenum_{i} E_{ij}(\vecx) \wavenum_{j} \kernel{u}(\vecx)
    \right) - \imag \left(
        \left[
            E_{ij}(\vecx) \wavenum_{j} \kernel{u}(\vecx)
        \right]_{,i}
        + \wavenum_{i} E_{ij}(\vecx) \kernel{u}_{,j}(\vecx)
    \right) &=
    -\omega^2 \rho(\vecx) \kernel{u}(\vecx) \label{eq:it.scalar.strong.general}
\end{align}

\paragraph{Isotropic modulus} i.e. \(E_{ij}(\vecx) = E(\vecx) \delta_{ij}\):
%
\begin{align}
    \left(
        \left[ E(\vecx) \kernel{u}_{,i}(\vecx) \right]_{,i}
        - \wavenum_{i} E(\vecx) \wavenum_{i} \kernel{u}(\vecx)
    \right) - \imag \left(
        \left[
            E(\vecx) \wavenum_{i} \kernel{u}(\vecx)
        \right]_{,i}
        + \wavenum_{i} E(\vecx) \kernel{u}_{,i}(\vecx)
    \right) &=
    -\omega^2 \rho(\vecx) \kernel{u}(\vecx) \label{eq:it.scalar.strong.iso}
\end{align}

\paragraph{Piecewise constant modulus} i.e. \(E_{ij,k}(\vecx) = 0\):
%
\begin{align}
    \Bigl(
        E_{ij}(\vecx) \kernel{u}_{,ji}(\vecx)
        - \wavenum_{i} E_{ij}(\vecx) \wavenum_{j} \kernel{u}(\vecx)
    \Bigr) - \imag \Bigl(
        E_{ij}(\vecx) \wavenum_{j} \kernel{u}_{,i}(\vecx)
        + \wavenum_{i} E_{ij}(\vecx) \kernel{u}_{,j}(\vecx)
    \Bigr) &=
    -\omega^2 \rho(\vecx) \kernel{u}(\vecx) \label{eq:it.scalar.strong.pc}
\end{align}

\paragraph{Piecewise constant isotropic modulus}
%
\begin{align}
    E(\vecx) \Bigl(
        \kernel{u}_{,ii}(\vecx)
        - \wavenum_{i} \wavenum_{i} \kernel{u}(\vecx)
    \Bigr) - \imag E(\vecx) \Bigl(
        \wavenum_{i} \kernel{u}_{,i}(\vecx) + \wavenum_{i} \kernel{u}_{,i}(\vecx)
    \Bigr) &=
    -\omega^2 \rho(\vecx) \kernel{u}(\vecx) \label{eq:it.scalar.strong.pc-iso}
\end{align}


%%%%%%%%%%%%%%%%%%%%%%%%%%%%%%%%%%%%%%%%%%%%%%%%%%%%%%%%%%%%
\subsubsection{Derivation of weak forms} \label{sec:it.scalar.weak-derivation}

The weak form is obtained by multiplying the strong form (\ref{eq:it.scalar.strong.general}) with the complex conjugate \(\bar{v}\) of a test function \(v\) and integrating over the unit cell. The test function is periodic on the unit cell. We derive the weak form for the general case first, then specialize it to the other three cases (\ref{eq:it.scalar.strong.iso}-\ref{eq:it.scalar.strong.pc-iso}). For conciseness, we omit the function notation ``\((\vecx)\)'', but it is to be understood that the trial and test functions \(\kernel{u}\) and \(v\), as well as the modulus \(E_{ij}\) and density \(\rho\), are functions of position.

\paragraph{General case}
%
\begin{align}
    \int_{\Omega} {
        \left(
            \left[ E_{ij} \kernel{u}_{,j} \right]_{,i}
            - \wavenum_{i} E_{ij} \wavenum_{j} \kernel{u}
        \right) \bar{v} \dx
    } - \imag \int_{\Omega} {
        \left(
            \left[
                E_{ij} \wavenum_{j} \kernel{u}
            \right]_{,i}
            + \wavenum_{i} E_{ij} \kernel{u}_{,j}
        \right) \bar{v} \dx
    } &=
    -\omega^2 \int_{\Omega} {
        \rho \kernel{u} \bar{v} \dx
    } \nonumber \\
    %
    \int_{\Omega} {
        \left[ E_{ij} \kernel{u}_{,j} \right]_{,i} \bar{v} \dx
    } - \int_{\Omega} {
        \wavenum_{i} E_{ij} \wavenum_{j} \kernel{u} \bar{v} \dx
    } - \imag \int_{\Omega} {
        \left[ E_{ij} \wavenum_{j} \kernel{u} \right]_{,i} \bar{v} \dx
    } - \imag \int_{\Omega} {
        \wavenum_{i} E_{ij} \kernel{u}_{,j} \bar{v} \dx
    } &=
    -\omega^2 \int_{\Omega} {
        \rho \kernel{u} \bar{v} \dx
    } \nonumber \\
    %
    \int_{\Omega} {
        \left(
            \left[ E_{ij} \kernel{u}_{,j} \bar{v} \right]_{,i}
            - E_{ij} \kernel{u}_{,j} \bar{v}_{,i}
        \right) \dx
    } - \int_{\Omega} {
        \wavenum_{i} E_{ij} \wavenum_{j} \kernel{u} \bar{v} \dx
    } & \nonumber \\
    - \imag \int_{\Omega} {
        \left(
        \left[ E_{ij} \wavenum_{j} \kernel{u} \bar{v} \right]_{,i}
        - E_{ij} \wavenum_{j} \kernel{u} \bar{v}_{,i}
        \right) \dx
    } - \imag \int_{\Omega} {
        \wavenum_{i} E_{ij} \kernel{u}_{,j} \bar{v} \dx
    } &=
    -\omega^2 \int_{\Omega} { \rho \kernel{u} \bar{v} \dx } \nonumber \\
    %
    \int_{\partial \Omega} {
        E_{ij} \kernel{u}_{,j} \bar{v} n_i \ds
    } - \int_{\Omega} {
        E_{ij} \kernel{u}_{,j} \bar{v}_{,i} \dx
    } - \int_{\Omega} {
        \wavenum_{i} E_{ij} \wavenum_{j} \kernel{u} \bar{v} \dx
    } & \nonumber \\
    - \imag \int_{\partial \Omega} {
        E_{ij} \wavenum_{j} \kernel{u} \bar{v} n_i \ds
    } + \imag \int_{\Omega} {
        E_{ij} \wavenum_{j} \kernel{u} \bar{v}_{,i} \dx
    } - \imag \int_{\Omega} {
        \wavenum_{i} E_{ij} \kernel{u}_{,j} \bar{v} \dx
    } &=
    -\omega^2 \int_{\Omega} { \rho \kernel{u} \bar{v} \dx } \nonumber
    %
    \intertext{The surface integrals are identically 0 because of the periodicity of \(\kernel{u}\) and \(v\) and can be dropped, leading to the transformed weak for for the general case:}
    %
    - \int_{\Omega} {
        E_{ij} \kernel{u}_{,j} \bar{v}_{,i} \dx
    } - \int_{\Omega} {
        \wavenum_{i} E_{ij} \wavenum_{j} \kernel{u} \bar{v} \dx
    } + \imag \int_{\Omega} {
        E_{ij} \wavenum_{j} \kernel{u} \bar{v}_{,i} \dx
    } - \imag \int_{\Omega} {
        \wavenum_{i} E_{ij} \kernel{u}_{,j} \bar{v} \dx
    } &=
    -\omega^2 \int_{\Omega} { \rho \kernel{u} \bar{v} \dx } \nonumber
\end{align}


%%%%%%%%%%%%%%%%%%%%%%%%%%%%%%%%%%%%%%%%%%%%%%%%%%%%%%%%%%%%
\subsubsection{Transformed weak forms} \label{it.scalar.weak}

\paragraph{General case}
%
\begin{align}
    - \int_{\Omega} {
        E_{ij} \kernel{u}_{,j} \bar{v}_{,i} \dx
    } - \int_{\Omega} {
        \wavenum_{i} E_{ij} \wavenum_{j} \kernel{u} \bar{v} \dx
    } + \imag \int_{\Omega} {
        E_{ij} \wavenum_{j} \kernel{u} \bar{v}_{,i} \dx
    } - \imag \int_{\Omega} {
        \wavenum_{i} E_{ij} \kernel{u}_{,j} \bar{v} \dx
    } &=
    -\omega^2 \int_{\Omega} { \rho \kernel{u} \bar{v} \dx }
    \label{eq:it.scalar.weak.general}
\end{align}

\paragraph{Isotropic modulus} i.e. \(E_{ij}(\vecx) = E(\vecx) \delta_{ij}\)
%
\begin{align}
    - \int_{\Omega} {
        E \kernel{u}_{,i} \bar{v}_{,i} \dx
    } - \int_{\Omega} {
        \wavenum_{i} E \wavenum_{i} \kernel{u} \bar{v} \dx
    } + \imag \int_{\Omega} {
        E \wavenum_{i} \left(
            \kernel{u} \bar{v}_{,i} - \kernel{u}_{,i} \bar{v}
        \right) \dx
    } &=
    -\omega^2 \int_{\Omega} { \rho \kernel{u} \bar{v} \dx }
    \label{eq:it.scalar.weak.iso}
\end{align}

\paragraph{Piecewise constant modulus} Because the weak form (\ref{eq:it.scalar.weak.general}) does not involve derivatives of the modulus, the weak forms for piecewise constant modulus and piecewise constant isotropic modulus are the same as (\ref{eq:it.scalar.weak.general}-\ref{eq:it.scalar.weak.iso}), respectively.



%%%%%%%%%%%%%%%%%%%%%%%%%%%%%%%%%%%%%%%%%%%%%%%%%%%%%%%%%%%%
\subsection{Elasticity equation} \label{sec:it.vector}

%\input{it.vector}


\end{document}